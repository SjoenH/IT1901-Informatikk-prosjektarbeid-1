\documentclass[12pt,a4paper,norsk]{article}
\usepackage[norsk]{babel}
\usepackage[utf8]{inputenc}
\usepackage{parskip}
\usepackage{natbib}
\usepackage{graphicx}
\usepackage{hyperref}

%\usepackage[toc,page]{appendix} %For appendix og toc blargh.

\usepackage{pdfpages} %For å legge til vedlegg som pdf-filer

\usepackage{dirtytalk}
\usepackage{changepage}

\usepackage{cleveref}

\usepackage{nicefrac}

\usepackage{attachfile}

% \usepackage{chicago} %Nope!
%\bibliographystyle{norchicago} %Noooope!

\hypersetup{
    colorlinks,
    citecolor=black,
    filecolor=black,
    linkcolor=black,
    urlcolor=black
}

\setcitestyle{square}

\title{IT1901 \\ Prosjektrapport}
\author{Gruppe 08\\
    755200, 741068, 757688,\\
    757676, 755591, 757787,\\
    757701, 757734, 757770\\ \\ \\
    Antall ord: 5998 \\ \\}

\renewcommand{\contentsname}{Innhold}
\renewcommand\refname{Referanser}

\usepackage{cleveref} %must be loaded last of the packages that modify LATEX’s referencing system.
\begin{document}
  \pagenumbering{gobble}
  \maketitle
  \newpage
  \pagenumbering{arabic}
  \tableofcontents
  \newpage
  \section{Introduksjon}

Denne oppgaven er gitt i emnet IT1901 - Informatikk prosjektarbeid 1, ved NTNU, høsten 2015. Målet med prosjektet er å gi et innblikk i hvordan det er å jobbe med større gruppeprosjekter og arbeidsmetoder hvor en tredjepart er med i prosessen. I dette prosjektet er tredjeparten en kunde. Prosjektet skal gi erfaringer i hvordan gruppeprosjekter gjennomføres, programmeringsverktøy, arbeidsmodeller og den smidige arbeidsmetoden \textit{Scrum}.

\subsection{Problemstilling}

Gruppen skal utvikle et administrasjonssystem for NTNUI Koiene. Systemet skal være et tillegg til NTNUIs eksisterende nettsider og gjøre det enklere å drifte koiene. Det skal eksempelvis være mulig for en bruker å rapportere vedstatus og ødelagte ting, samt få beskjed om å ta med seg ting til en reservert koie. Medlemmer av Koiestyret skal kunne se disse rapportene og bruke de til å planlegge veddugnader og reparasjoner.

Programmet skal utvikles etter kundens ønsker og behov. Kunden har tildelt en rekke brukerhistorier som skal være oppfylt i det ferdige produktet.

\section{Gruppen og rutiner}
\subsection{Rutiner og timeplan}

Gruppens første arbeidsoppgave var å lage en timeplan for prosjektet. Gruppen bestemte seg for å møtes og jobbe sammen 10 timer i uka. (Se vedlegg \cref{app:motereferat} for detaljer).

\subsection{Gruppen og ansvarsfordeling}
Gruppen bestemte seg tidlig for at alle skulle jobbe med forskjellige arbeidsoppgaver, men ikke låses i fastsatte roller. Dette for å dele kompetansen og la alle få erfaring i de forskjellige aspektene ved prosjektet. Sånn får man gjort seg mindre avhengig av enkeltpersoner og får aktivisert gruppemedlemmer så mye som mulig.

  \section{Arbeidsprosess}
  \subsection{Scrum i teori}
  \textit{Scrum} er et rammeverk for å utvikle eller videreutvikle produkter, og er veldig populært innen software-utvikling. \textit{Scrum} er en smidig metode, og har som alle smidige metoder fokus på å respondere på endringer og samarbeid med kunden i løpet av prosjektets gang. [\cite{agilemanifesto}]

  I et utviklingsprosjekt har man gjerne to parter: kunde og utvikler-team. I \textit{Scrum} har man et \textit{Scrum}-Team med en produkteier, en \textit{Scrum}-master og utviklere (se  \cref{subsec:scrumroller} om \textit{Scrum}-roller). Produkteier er kundens representant i \textit{Scrum}-teamet. Produkteier jobber i samarbeid med utvikler-gruppen, ikke bare i planleggingsfasen av prosjektet, men også regelmessig under utvikling.
  [\cite{scrumguides}]
  \subsubsection{Produkt-backloggen}
  Produkteier stiller krav til prosjektet i form av en liste med brukerhistorier. Brukerhistorier er en kort forklaring av hva som skal kunne gjøres med et mål.  Brukerhistorier skrives ofte på formen “Som X skal jeg kunne Y for å Z”, hvor X beskriver brukertypen, Y handlingen brukeren skal kunne utføre og Z målet brukeren har med å utføre handlingen. \cite[side 9]{kniberg} Denne listen med brukerhistorier kalles for produkt-backloggen.
  %Hva menes med Mål?

    Backloggens brukerhistorier får en ID og et kort beskrivende navn. De blir vurdert fra viktigst til minst viktig, på bakgrunn av produkteiers prioriteringer og ønsker, men med eventuelle innspill fra utvikler-teamet.
  \subsubsection{Sprintplanleggingsmøtet}
  Utviklingsfasen deles opp i sykler, og blir i \textit{Scrum} kalt sprinter. I begynnelsen av hver sykel har man i \textit{Scrum} et sprintplanleggingsmøte. Deltakerne på møtet er ikke bare medlemmer av utviklingsteamet med \textit{Scrum}-master, men også produkteier. Formålet med møtet er å forberede og justere produkt-backloggen og planlegge neste sprint.

    \textit{Scrum}-teamet avgjør hvor lang sprinten skal være og finner ut hvilke ressurser man har på denne sprinten. Gruppen avtaler tidspunkt for når på dagen gruppen skal ha det daglige \textit{Scrum}-møtet, når retrospektiv-møtet for sprinten skal være og når demonstrasjonen for produktet skal være \cite[side 16]{kniberg}.

  \subsubsection{Release-plan}
  I noen prosjekter kan det være hensiktsmessig å planlegge lengre frem i tid enn kun for den neste sprinten. Man lager en releaseplan for hele prosjektet. Hensikten med denne er å gjøre det samme som i sprintplanleggingsmøtet, men for alle brukerhistorier. Etter at en sprint er ferdig har man et sprintplanleggingsmøte der man justerer release-planen utifra eventuelle endringer og hvordan sprinten faktisk gikk. \cite[side 95 - 101]{kniberg}

  \subsubsection{Daglig Scrum}
  I \textit{Scrum} har man et daglig møte der hver person svarer på de tre spørsmålene:

  \begin{itemize}
        \item[1.] Hva gjorde jeg sist som hjalp gruppen å nå sprintmålet?
        \item[2.] Hva skal jeg gjøre i dag for å hjelpe gruppen å nå sprintmålet?
        \item[3.] Ser jeg noen eventuelle utfordringer for meg eller gruppen, som kan hindre oss å nå sprintmålet?
    \end{itemize}

    \textit{Scrum}-møtet skal være kort, med fokus på å få hjelp om man har noen utfordringer.

  \subsubsection{Retrospektiv-møtet}
      Retrospektive møter en en viktig del av \textit{Scrum}. Retrospektiv-møtet er gruppens mulighet til å forbedre prosesser og arbeidsrutiner. Retrospektiver har man på slutten av hver sprint, og både utviklingsgruppen, \textit{Scrum}-master og produkteier deltar. På møtet diskuterer man og svarer på følgende spørsmål:

  \begin{itemize}
        \item[1.] Hva gikk bra i sprinten?
        \item[2.] Hva gikk mindre bra i denne sprinten?
        \item[3.] Hva kan forbedres til neste sprint?
    \end{itemize}

    Det siste punktet er hovedfokuset i møtet. Det er viktig å komme med konkrete eksempler slik at man er helt sikker på hva det er man kan endre til neste gang. Retrospektiv skal være den tryggeste måten for gruppen å få frem sin mening, det er derfor viktig at alle i gruppen blir hørt og at ingen skal føle at de ikke kan si sin mening. En måte å få til dette er å la én etter én få si sin mening om sprinten, uten at andre avbryter eller kommenterer det som blir sagt. Etter at alle har fått sagt det de vil, så kan man diskutere i gruppen hva som er viktigst å fokusere på og forbedre neste sprint.

    Disse møtene er viktig fordi det alltid er rom for forbedring. Hvis man ikke har et oppsatt møte der man tar opp hva som har skjedd i sprinten så vil man gjøre akkurat de samme feilene i den neste. Dette gir gruppen muligheten til å forbedre seg utover prosjektet.
    \cite[side 82 - 88]{kniberg}
  \subsubsection{Scrum-roller}\label{subsec:scrumroller}
  \textbf {Produkteier}
    \par Produkteier er kundens og eventuelle andre interessenter sin representant. Vedkommende er ansvarlig for å oppdatere  produktbackloggen og for å kommunisere til utviklerne hva brukerhistoriene innebærer.
    [\cite{scrumguides}]

    \bigskip \noindent \textbf{\textit{Scrum}-master}
    \par En \textit{Scrum}-master er en tilrettelegger for et produktutviklingsteam som bruker \textit{Scrum} som utviklingsmetodikk, som gir teamet muligheter for selvorganisasjon og håndtering av plutselige endringer under prosjektet. I tillegg til det bruker han/hun \textit{Scrum} for å effektivisere hele arbeidsprosessen og er ansvarlig for at \textit{Scrum} blir brukt som tiltenkt.

    \begin{itemize}
    \item[-] Hjelper med å rekke sprintmålene.
    \item[-] Støtter produkteieren med å oppdatere backloggen ved potensielle endringer underveis slik at teamet kan reagere på de.
    \item[-]  Hjelper teamet med å finne en eksplisitt definisjon for “done” med vurdering av innspill fra kunden.
    \item[-]  Lærer bort \textit{Scrum}-teknikker til teamet for å sikre at produktet leveres med høy kvalitet.
    \item[-]  Hjelper teamet med å jobbe mer selvstendig.
    \item[-]  Hjelper teamet med å identifisere og unngå eller fjerne potensielle og faktiske kilder for forstyrring og hindring for både interne og eksterne kilder.
    \end{itemize}[\cite{scrummaster}][\cite{scrummasterrolle}]

    \bigskip \noindent \textbf{Utviklere}
        Medlemmene av utvikler-gruppen har ansvar for å gjøre produkt-backloggen om til et fungerende produkt. Utviklere er ikke delt inn i delgrupper. Medlemmer kan ha forskjellige fokusområder, men alle har like mye ansvar for produktets kvalitet. Utviklere ledes ikke av \textit{Scrum}-master, men får hjelp til å være selvorganiserende.

  \subsubsection{Task-board}
  Task-boardet er et verktøy for å gjøre hele arbeidsprosessen mer oversiktlig og interaktiv. Man kan se hvilke brukerhistorier og deloppgaver som må gjøres for sprinten, om de er delegert, om de er ferdige og hvilken prioritet de har. Figur \ref{fig:task-board} viser et eksempel-task-board.

    \begin{figure}[h!]
    \includegraphics[width=\linewidth]{img/task-board.png}
      \caption{Eksempel på task-board med brenndiagram}
      \label{fig:task-board}
  \end{figure}

    \textit{Task-boardet} er delt inn i 4 kolonner: \textit{Not checked out}, \textit{Checked Out}, \textit{Done} og \textit{Sprintmål}.

    \textit{Not checked out}: Her er brukerhistoriene ingen har startet på enda. Brukerhistoriene er sortert etter viktighet. De viktigste står på toppen. Historiene er splittet i flere mindre oppgaver.

    \textit{Checked out}: Her er brukerhistoriene, deloppgavene og “ikke planlagte ting” (se \ref{subsec:ikkeplanlagt}) som allerede jobbes med.

    \textit{Done}: Når en brukerhistorie er ferdig, ender den og alle deloppgaver her. Alle som jobber med prosjektet vet da at denne historien oppfyller kravene som er nødvendig for at den kan kalles for “ferdig/done”.

    \textit{Sprintmål}: Her vises sprintmålet og en seksjon som er delt i tre: Brenndiagram, Ikke planlagte ting og de neste historiene i produkt-backloggen.

    \paragraph{Brenndiagram:}
    Et todimensjonalt diagram der x-aksen viser datoene mellom sprintstarten og sprintslutten. Y-aksen viser arbeidet som er igjen i storypoints. En tegner en graf ut fra hvor man er i sprinten (dato) og hvor mange deloppgaver/brukerhistorier som er blitt fullført denne dagen. Dersom tidsestimatene er riktige ender man opp med en graf som går fra diagrammets venstre topp til høyre bunn i en nesten rett linje.

    \paragraph{Ikke planlagte ting:}\label{subsec:ikkeplanlagt}
    Her finnes deloppgaver som dukker opp underveis. Disse var ikke planlagt fra starten, men må implementeres for at produktet skal fungere. Ved for dårlig planlegging, kan dette føre til en flom av ikke planlagte elementer. Arbeidsmengden vil da øke raskere og det ender med en for sakte synkende graf.

    %Høres ut som.. hvordan tolke brenndiagrammet...
    %Trenger vi dette? TODO: make up our hive-mind.
    Faresignaler fra brenndiagrammet kan være: Hvis grafen synker for fort, har brukerhistoriene tatt mindre tid enn forventet, dette skyldes dårlig estimering. Synker grafen for sakte har estimatene vært for lave. Disse problemene løses motsatt, synker grafen for fort flyttes brukerhistorier fra \textit{neste historier} og inn i sprint, synker den for sakte flyttes brukerhistorier inn i \textit{neste historier}.

    \paragraph{Neste historier:}
    Her er alle brukerhistoriene som ikke er implementert enda og ikke har fått plass i den nåværende sprinten.

    \cite [side 72 -78]{kniberg}


  \subsubsection{Utviklingsverktøy}

%\textbf{PyCharm}
%\par De fleste på gruppen brukte PyCharm for å jobbe med pythonfilene. PyCharm er en IDE som er mye brukt i programutvikling siden det har mange flere funksjoner enn en vanlig teksteditor som gjør jobben enklere. I tillegg er det mulig å teste koden med en gang i konsollen, samt at GIT kan brukes i selve programmet for å holde prosjektet oppdatert.

\bigskip \noindent \textbf{GIT og Stash}
\par For å holde orden på versjoner og endringer i prosjektet brukte vi GIT koblet opp til Stash for å forsikre oss om at ingen endringer og oppdateringer gikk tapt. På Stash hadde vi et repository, dette er et område der all koden til prosjektet blir lagret. Repositoriet var delt opp i branches, der forskjellige versjoner av koden lå lagret adskilt. Master var selve hovedbranchen hvor nåværende produkt lå. Hver gang noen skulle lage en ny endring i produktet, så ble det laget en ny sidebranch, denne ble så merget med masterbranchen når endringen var ferdig. Dette gjør at det er mulig at flere personer jobber med samme kode på forskjellige maskiner samtidig, uten at det blir gjort feil.

I tillegg til at man kan ha flere branches, vil man trenge godkjenning av andre gruppemedlemmer for å merge en ny branch inn i master. Slik kan man være sikker på at feil i koden blir plukket opp tidlig. Alle endringer som har blitt gjort på prosjektet blir også lagret, så om noe virkelig går galt er det mulig å gå tilbake til en tidligere versjon og gjøre ting på nytt.


  \subsection{Scrum i praksis}
    Det vil omtrent aldri være mulig å overføre teori direkte til praksis. Kniberg skriver i introen til \textit{Scrum and XP from the trenches} \cite[side 2]{kniberg}:

    \bigskip\begin{adjustwidth}{2.5em}{0pt}
    \say {\textit{The strength and pain of Scrum is that you are forced to adapt it to your specific situation}}
    \end{adjustwidth}
    \bigskip Gruppens største utfordring med å tilpasse \textit{Scrum} var på grunn av forskjellige timeplaner og mangel på et fast og privat rom. I motsetning til en prosjektgruppe i et konsulentfirma ville vi ikke ha muligheten til å møtes hver dag i uka på samme sted.

  \subsubsection{Sprintlengde}
  Gruppen bestemte at 2 uker lange sprinter var praktisk fordi vi kunne møte med veileder midt i sprinten og produkteier på slutten. Korte sprinter ville gjøre det mulig for oss å ha retrospektive møter ofte og få hyppige tilbakemelding fra produkteier. På slutten av hver sprint var planen å teste funksjonene som hadde blitt implementert med produkteier. Dette i form av en brukbarhetstest med scenario-tester.

    Med to uker lange sprinter ville vi oppnå 4 sprinter, med avslutning 4. november. Ettersom innlevering av rapport hadde frist 12. november ville vi ha mulighet etter siste sprint til å jobbe med ferdigstilling av rapporten.

    De timene vi planla å jobbe hver uke ble delt inn i økter for å estimere arbeidskapasitet per sprint. Denne enheten var også den vi brukte når vi skulle estimere hvor mange storypoints en brukerhistorie krevde - 1 økt = 1 storypoint. En person ville ha 5 økter: mandag: 1, torsdag 2, fredag: 2 per uke. Siden vi var 9 personer, ville vi ha sprinter med 9 (personer) x 5 (økter pr uke) x 2 (uker) = 90 økter pr. sprint. Vi anslo at av 90 teoretiske økter i løpet av en sprint ville ca. 50 økter bli brukt på brukerhistorier. Tallet er lavere siden noen økter går til møter, mens andre går til rapportskriving, oppsett av utviklermiljø, tid til å sette seg inn i rammeverk eller forbereding til demonstrasjoner og tester. Vi antok at mye av tiden av vår første sprint kom til å gå til planlegging av prosjektet og sette opp utviklermiljø, og anslo at produktive økter kom til å være på ca. 25 økter. Sprint 2,3 og 4 anslo vi til 50 arbeidsøkter, fordi vi antok at vi da kom til å være satt opp med prosjektet, kommet inn i bruk av \textit{Scrum} og klare til å bruke mer tid på utvikling, og mindre tid på administrativt arbeid.

  \subsubsection{Brukerhistorier}
  Produkteier rangerte brukerhistoriene og ga dem lav, høy eller vanlig prioritet. Deretter måtte vi rangere historiene igjen, for å finne hvilke vi skulle implementere først av de med høyest prioritet. Noen historier så ut til å være avhengige av at andre var ferdige, og måtte nødvendigvis havne i tilsvarende rekkefølge. På grunn av estimater på noen historier måtte vi også gjøre en brukerhistorie med lavere prioritet før en med høyere prioritet, fordi det var den eneste vi fikk plass til i sprinten.


    Når vi skulle estimere hvor mange arbeidsøkter en brukerhistorie kom til å ta, brukte vi \textit{planning poker} på hver brukerhistorie. \textit{Planning poker} utføres ved at hvert gruppemedlem får en kortstokk med 13 kort, med verdiene 0, \nicefrac{1}{2}, 1, 2, 3, 5, 8, 13, 20, 40, 100, ?, “Pause”. Alle velger ett av kortene som representerer estimert tid for å løse brukerhistorien. Har man bestemt seg for et kort legges det på bordet med baksiden opp. Når alle har et kort foran seg snur man dem. Dersom det oppsto store avvik i valg, skal man diskutere hvorfor estimatene blir forskjellige. Deretter skal brukerhistorien estimeres igjen for å se om gruppen blir mer enige om estimatene \cite[side 38-40]{kniberg}.
    Det var nyttig. Vi fikk avdekket forskjellige oppfatninger i gruppen av hva slags tekniske implikasjoner historien innebar. Når en historie ga vidt forskjellige verdier, så kunne de som mente at historien trengte få økter, få si hvorfor de mente den ikke trengte så mange og motsatt. Det kom opp mange gode forslag på hvordan brukerhistorier kan løses på et teknisk nivå. Noen i gruppen visste allerede litt hvordan muligheter rammeverket ga, og hadde allerede idéer på løsninger av problemer. Når vi tok en ny runde \textit{planning poker} på den samme brukerhistorien fikk vi bedre resultat og kunne fastsette antall økter.

  \subsubsection{Vårt Task-board}
  Gruppen bestemte seg for å organisere task-boardet via \textit{Trello.com}. \textit{Trello} er et nett-verktøy som kan brukes som et digitalt task-board. Alle på gruppen ble medlem av \textit{Trello}, og vi lagde et team som alle ble medlem av. Vi inviterte også produkteier til å bli medlem, så produkteier kunne ha oversikt over hvordan vi lå an, og skrive inn prioriteringer på brukerhistoriene.

    I \textit{Trello} kan man lage forskjellige brett, noe vi gjorde for hver sprint. På hvert respektive brett la vi til lister med kort, der hvert representerte en brukerhistorie. For hvert brett ble det laget lister for “Not checked out”, “Checked out”, “Andre oppgaver” og “Done”. Etter releaseplanleggingen ble brukerhistoriene lagt inn som kort i “Not checked out” for alle sprintene.

    Når noen skulle begynne på en brukerhistorie, ble det tilsvarende kortet flyttet over til checked out. \textit{Trellobrukerne} til de som jobbet med brukerhistorien ble knyttet opp mot kortet, så det ble lett å se hvem som jobbet med hva.

    Utvidelsen \textit{Burndown for Trello} lot oss legge inn estimert tidsbruk og storypoints på et kort, hvor mye tid som faktisk ble brukt på hvert kort, og automatisk generere brenndiagram basert på de to. (Se vedlegg ~\ref{app:taskboard} for forklaring på hvordan vi brukte \textit{Trello} som task-board.)
  \subsubsection{Våre Retrospektiver}

\bigskip \noindent \textbf{Retrospektiv Sprint 1}
\par Dagen etter første sprint hadde gruppen sitt første retrospektivt. Gruppen  diskuterte hvordan første sprint hadde gått. Gruppemedlemmene fikk hver sin tur til å si: “Hva som gikk bra”, “Hva som gikk dårlig” og “Hva vi kan forbedre” (se vedlagt møtereferat: ~\cref{app:sprint1}).

\paragraph{Hva gikk bra:}
I den første sprinten var gruppen mer effektiv enn vi hadde regnet med, dette resulterte i at gruppen fikk fullført flere brukerhistorier enn antatt. Samarbeidet i gruppen fungerte veldig bra, spesielt var gruppemedlemmene flinke til å hjelp hverandre. Fordelingen av arbeidsoppgaver gikk lett, gruppen delte seg opp i mindre grupper som tok ansvar for hver sine arbeidsoppgaver. Gruppemedlemmene var også flinke til å dele kunnskapen med hverandre. En måte gruppen delte kunnskap var ved bruk av \textit{parprogrammering} \cite[side 17]{dyba}, der små grupper kodet sammen. Et av gruppemedlemmene skrev koden mens resten hjalp den som kodet. Denne metoden lot de som ikke var helt komfortable med teknologiene som ble brukt, bli bedre kjent mens de kunne støtte seg på andre gruppemedlemmer. Gruppens medlemmer var flinke til å tilegne seg ny kunnskap på egenhånd.


\paragraph{Hva gikk ikke bra:}
På første møte bestemte gruppen at det ukentlig skulle være sosiale samlinger, dette var for å bli bedre kjent og for å få en pause i lange arbeidsøktene. Den sosiale samlingen ble bestemt til å spise pizza på fredager. Dette ble ikke gjennomført den første sprinten
Gruppen hadde litt fravær i løpet av sprinten.
Gruppen satte av tid for å sette opp utviklingsmiljøene og til å lære verktøyene som skulle brukes i prosjektet. Til tross for dette oppsto det problemer med flere av utviklingsmiljøene, som førte til at ting tok ekstra tid.

På grunn av problemene med utviklingsmiljøene og mangel på oppgaver, var det til tider vanskelig å sysselsette alle gruppemedlemmene.

Fordeling av arbeidsoppgaver på grupper førte til at ikke alle gruppemedlemmene fikk satt seg inn i de samme teknikkene.
Gruppen underestimerte arbeidet gruppen fikk gjort.


\paragraph{Hva kan forbedres:}
Inkludere gruppen i flere av oppgavene, noe som fører til at alle får kodet og jobbet med rapporten. Samt passe på at alle har en arbeidsoppgave.
Gruppen må passe på å ta jevnlige pauser, og ta tid til sosialisering.
Loggføring av arbeid og timer på arbeidsoppgaver må skjerpes for å gjøre det lettere å lage brenndiagram.
Det ble bestemt å lage en risikoplan, slik at gruppen var forberedt på uforutsigbare hendelser.


\bigskip \noindent \textbf{Retrospektiv Sprint 2}
\par Gruppen hadde retrospektiv dagen etter demo for sprint to, hvor gruppen gikk igjennom de samme tre spørsmålene som i første retrospektiv, (se vedlagt møtereferat: ~\cref{app:sprint2}).

\paragraph{Hva gikk bra:}
Noe gruppen skulle forbedre fra forrige sprint var å gi alle i gruppen en oversikt over prosjektet. Gruppen var flinkere til å inkludere flere medlemmer i flere oppgaver, som førte til bedre forståelse av kode og bedre oversikt over prosjektet i helhet for alle gruppemedlemmer. En annen konsekvens at alle ble inkludert i flere oppgaver var at arbeidsoppgavene ble bedre justert til gruppemedlemmene.

I løpet av sprinten hadde gruppen fokus på lagbygging. Vi tok en pause fra arbeid og spiste pizza sammen hver fredag, en “tradisjon” vi fortsatte med. Innimellom på møtene prøvde vi også å gjøre noe morsomt sammen, som å se videoer på youtube i pausene.
Generelt er inntrykket at vi jobber bra som en gruppe og ligger godt an.
Vi fokuserte mer på administrative oppgaver og rapportskriving. Risikoanalyse og testplanen ble utbedret.

\paragraph{Hva gikk ikke bra:}
I slutten av sprinten fant vi ut at arbeidsmengden var for stor. Det gjorde at vi måtte overføre noen deloppgaver fra brukerhistorien til neste sprint.
Et problem vi merket denne sprinten var at vi fikk mindre ressurser å jobbe med. Det var større fravær enn tidligere, i tillegg kom gruppemedlemmer oftere for seint, som gjorde at det tok lengre tid å starte møtene.

\paragraph{Hva kan forbedres:}
Siden mange var borte denne sprinten så ble gruppen enig om å bli bedre på å finne seg arbeidsoppgaver å gjøre hjemme ved fravær.
I tillegg skal vi ha en person som følger opp de som er borte/seine og ha 1-1 samtaler med dem for å finne bedre ut hvorfor de er seine og hvordan det kan unngås.


\bigskip \noindent \textbf{Retrospektiv Sprint 3}
\par Etter tredje sprinten hadde vi et retrospektivt møte før vi begynte neste sprint, (se vedlagt møtereferat: ~\cref{app:sprint3}). Da gjorde vi det samme som i de forrige retrospektivmøtene.

\paragraph{Hva gikk bra:}
Flere i gruppen har tatt mer ansvar for eget arbeid, og finner oppgaver å gjøre. Spesielt har det gått bra å jobbe hjemmefra når noen i gruppen har vært syk. Dette viser god forbedring fra forrige sprint da vi ville bli bedre på akkurat dette, mye på grunn av personoppfølging.
Gjennom sprinten har gruppen vært effektiv med arbeidsoppgavene, det vises ved at oppgavene som var igjen fra forrige sprint, de fra denne sprinten og deler av neste sprint ble fullført. Gruppen har vært flink til å finne enkle løsninger på oppgavene.
%I tillegg til brukerhistorier og design har gruppen også gjort noe morsomt på nettsiden med noen gjemte funksjoner, også kalt “easter eggs”.

\paragraph{Hva gikk ikke bra:}
En av tingene som gikk dårlig i denne sprinten var estimater av tid på brukerhistoriene, vi regnet med å bruke lengre tid på dem. Dette var fordi ingen hadde nok erfaring med verktøyene som skulle brukes til å gjøre et mer riktig estimat, f.eks. for å lage kart på en nettside.
Det var dårlig fokus på å føre timer som ble brukt på de forskjellige oppgavene og å skrive logg for hva som skjedde på hvert møte.
Flere i gruppen ble syk, men en i gruppen tok alltid kontakt da noen var borte og ikke ga beskjed, slik at de hadde arbeidsoppgaver.
Mot slutten av sprinten var mye av arbeidet ferdig, da var det generelle inntrykket i gruppen at alt var ferdig. Det gjorde at man slappet litt mer av og ikke jobbet like hardt som ønskelig, iallefall siden det viste seg at det var problemer med noen av funksjonene som vi måtte fikse i neste sprint.
Når det kom til rapporten så mente flere i gruppen at det var vanskelig å vite hvordan rapporten skulle skrives, det førte til at deler av rapportskrivingen gikk sakte da man satt seg fast.

\paragraph{Hva kan forbedres:}
Vi har sett tidligere at en logg for hvert møte kan være bra for de fraværende så de får oversikt over hva som blir gjort når de er borte. Selv om kommunikasjonen har gått bra denne sprinten så er det fortsatt ønskelig å ha en logg til senere, så dette er et punkt gruppen må forbedre til neste sprint.
Et annet punkt gruppen kan bli bedre på er å holde de møtene vi har fokusert på de oppgavene som må gjøres, til tider hender det at fokuset drifter over på ting som ikke er så viktig. Siden det fortsatt er mye fravær og forsentkomminger så ville det vært bra for gruppen at alle blir flinkere til å komme til tide, kanskje ved å sette på 2 alamer på morgenen.
Noe som er veldig viktig er at man tester funksjoner bedre før man skriver at de er ferdig, da slipper gruppen å tro at alt er ferdig når det fortsatt er feil i funksjoner.


\bigskip \noindent \textbf{Retrospektiv Sprint 4}
\par Etter den tredje sprinten hadde vi også et retrospektivt møte før vi begynte neste sprint, (se vedlagt møtereferat: ~\cref{app:sprint4}). Da gjorde vi det samme som i de forrige retrospektivmøtene.

\paragraph{Hva gikk bra:}
I denne sprinten ble produktet ferdig, selv om det ikke var mange oppgaver igjen denne sprinten er gruppen fornøyd med å være ferdig. Dette har vi gjort ved å ha god arbeidsånd mot slutten av prosjektet.
Gruppen er fornøyd med hvor mye som er ferdig på rapporten, og det merkes at det har blitt gjort arbeid på den i tidligere sprinter også.

\paragraph{Hva gikk ikke bra:}
Gruppen var ikke forberedt på at det skulle skrives dokumentasjon på all koden, dette var ikke en del av planen vi la opp mot slutten av prosjektet.
Selv om vi har jobbet godt og fått gjort mye så har det til tider vært litt dårlig fokus og tull på møtene. En grunn til dette er at prosjektet går mot slutten og folk begynner å bli litt lei og slitne, spesielt når man sitter og skriver rapport i lengre perioder. Gruppen merket til tider at 4 timer i strekk var litt for lange arbeidsmøter, derfor ble de kuttet ned til 2 timer.
Et problem med å skrive rapport er at vi har skrevet over maksgrensen for ord, som kan føre til at det er noe irrelevant og duplisert arbeid.

\paragraph{Hva kan forbedres:}
En ting gruppen kunne forbedret var å involvere flere personer i rapportskriving tidligere, slik at det ikke ble like mye å gjøre mot slutten.
Gruppen skulle også klart å holde bedre fokus på møtene, for å være enda mer effektive.
En annen ting vi kunne gjort bedre var å endre arbeidsmetoden i denne sprinten, der det for det meste bare var rapportskriving. Fordi måten man jobber best med kode og rapport er ikke nødvendigvis det samme, f.eks. lengde på arbeidsøkter, og man kan gjøre mer hjemme og møtes færre ganger.

\subsubsection{Sprintgjennomgang}
\bigskip \noindent \textbf{Sprint 1}
\par Sprint 1 startet med et sprintplanleggingsmøte der det ble diskutert hva som  skulle gjøres i sprinten og hvilke brukerhistorier vi skulle utføre. Disse ble valgt etter hva kunden hadde satt opp som høy prioritet. (Ref: \cref{app:releaseP3}) Gruppen laget også en brukerhistorie, administrasjon av koiene på nettsiden, som ble satt til sprint 4 siden det ble antatt at den kom til å ta lang tid.

Halvparten av arbeidsøktene var satt til å sette opp Stash, installere og sette opp nødvendige utviklerverktøy. Administrasjonsdelen og starten av nettsiden ble også laget. Dette tok mindre tid enn estimert og det ble mer tid til brukerhistoriene.

Da brukerhistoriene ble startet, delte gruppen seg opp i to og gjorde en brukerhistorie hver. Begge brukerhistoriene handlet om å rapportering, \textit{ved} og \textit{ødelagte ting}. De var veldig like så alle på gruppen kunne lære seg det grunnleggende og hjelpe hverandre. Disse brukerhistoriene tok noe lenger tid enn estimert siden alle var nybegynnere. Dette jevnet seg ut etterhvert.

Den siste brukerhistorien i sprinten, \textit{koiestyret skal kunne ta ut rapporter av ødelagte ting}, tok mindre tid enn estimert, fordi det kun var å koble rapporterte ting til admin-panelet. Gruppen fikk derfor tid til overs og valgte å overføre brukerhistorien \textit{rapportere tekniske utbedringer på en koie} fra sprint 2 til 1 fordi denne hadde lavest prioritet.

Brenndiagrammet viser progresjonen for oppgavene i denne sprinten (se vedlegg, sprint1: \cref{app:brenn}.)

\bigskip \noindent \textbf{Sprint 2}
\par Sprinten startet med et sprintplanleggingsmøte. Vi gikk ut fra releaseplanen, men siden en av brukerhistoriene ble flyttet til sprint 1 ble det hentet en fra sprint 3 (se vedlagt releaseplan-v3: ~\cref{app:releaseP3}). Gruppen valgte brukerhistorien med likest tidsestimat til den som ble flyttet til sprint 1. Gruppens egenlagde brukerhistorie om \textit{administrasjon av koiene} ble implementert av seg selv da andre oppgaver ble gjort, den ble ikke arbeidet på og ble derfor fjernet. Gruppen valgte å lage enda en brukerhistorie, der nettsiden skulle designes med forside og navigering. Dette gjenspeiler målet for sprinten; \textit{Ha en fungerende nettside som kun mangler litt funksjonalitet}.

I denne sprinten handlet brukerhistoriene om direkte kontakt mellom koiestyret og brukere, og å ta ut rapporter for ved og teknisk stand. Her ble oppgavene fordelt på mindre grupper.
Dette gikk veldig greit å utføre, fordi det lignet oppgavene fra forrige sprint.

Den viktigste brukerhistorien i sprinten var å kunne ta ut vedrapporter og -prognoser for koiene. Den var litt vanskeligere, fordi man selv måtte definere hvor mye ved som gikk når koien ble brukt, og lage en graf ut av dette samt en prognose for når det blir tomt.

Gruppen ville fullføre brukerhistoriene som handlet om kommunikasjon mellom koiestyret og brukeren, derfor ble et epostsystem for å melde fra om ting som måtte tas med til og fra koier implementert.

Arbeidet med nettsiden kom ganske langt, det ble laget forside og en navigasjonsmeny der implementerte funksjonener ble lagt til. Nettsiden fikk derfor mye funksjonalitet veldig tidlig, blant annet å kunne melde inn ved, ødelagte ting og tekniske utbedringer fra brukeren, og at koiestyret kunne hente ut rapportene.

I slutten av sprinten ble det funnet noen bugs i vedprognosefunksjonen som det ikke ble tid til å fikse, disse ble overført til neste sprint.

Brenndiagrammet viser hvordan progresjonen for oppgavene var i denne sprinten (se vedlegg, sprint2: \cref{app:brenn}).

\bigskip \noindent \textbf{Sprint 3}
\par Sprinten startet med et sprintplanleggingsmøte. Vi gikk ut fra releaseplanen (se vedlegg releaseplan-v3: ~\cref{app:releaseP3}). Denne sprinten hadde kun 2 brukerhistorier, men tidsestimatene var veldig høye. Gruppen hadde også oppgaver fra forrige sprint, og en del arbeid med å få nettsiden til å se ferdig ut. Det ble også holdt en demonstrasjon av produktet.

Brukerhistorien \textit{oversikt over vedlikeholdsbehov utfra forventet levealder på konstruksjonen av koien} var det noe usikkerhet rundt. Måtte man gå gjennom alle koiene og finne ut når de ble bygget og oppusset, og når de må gjøres noe med? Etter å ha snakket med kunden kom man frem til en bedre løsning der man istedenfor legger inn når noe er reparert/satt inn nytt og hvor lenge disse vil vare, man vil da få varsel ved nødvendige utbedringer.

Brukerhistorien om kart over koiene med ved- og utstyrsstatus var forståelig, men å lage kartet tok litt tid siden det var noe nytt å sette seg inn i. Kartet ble laget med riktige posisjoner på koiene, man kan trykke på koiene og få opp link til vedstatus og utstyr. Her fikk vi en bug der man ikke kom til riktig koie på linkene, den ble ført over til sprint 4.

Begge brukerhistoriene tok mindre tid enn estimert, derfor bestemte gruppen seg for å hente og utføre den siste brukerhistorien fra sprint 4. Der skulle koiestyret få oversikt over overnattinger på koier i gitt område. Gruppen laget et script som hentet informasjon om overnattinger fra NTNUI sine nettsider og det ble deretter laget soner på kartet med koiene.

Det ble også jobbet med nettsiden denne sprinten, både med administrasjonsiden slik at den ble lettere å forstå og med å gjøre nettsiden mer brukervennlig.

Brenndiagrammet viser hvordan progresjonen for oppgavene var i denne sprinten (se vedlegg, sprint3: \cref{app:brenn}).

\bigskip \noindent \textbf{Sprint 4}
\par Sprinten startet med sprintplanleggingsmøte. Vi gikk ut fra releaseplanen. Brukerhistorien for denne sprinten hadde blitt overført til sprint 3, (se vedlagt releaseplan-v3: \cref{app:releaseP3}).
\par Oppgavene som Gjensto gikk ut på finpussing og testing av systemet generelt, som reflekteres i målet for sprinten: Få et fullstendig produkt, hvor produktets funksjoner oppleves intuitivt for brukeren.

Nettsiden ble endret på flere måter, på grunn av god tid ble kartene endret, fordi vi ville ha en finere løsning, og det var en bug med linkene på kartet som også måtte fikses. Det ble også endringer sidedesignet, som å gjøre navn på linker lettere å forstå. Etter å ha snakket med kunden ble posisjonen av kartet og informasjonen om overnattinger gjort mer oversiktlig.

Bortsett fra dette gikk tiden til rapportskriving, testing, skriving av dokumentasjon og utføring av akseptansetest med kunden.

Brenndiagrammet viser hvordan progresjonen for oppgavene var i denne sprinten (se vedlegg, sprint4: \cref{app:brenn}).


  \section{Risikoanalyse}
  Tidlig i prosjektet bestemte gruppen seg for å skrive en risikoanalyse for å se hvilke faktorer som kunne hindre fremgangen i gruppen. Alle risikofaktorene er tilgjengelig i tabellform under vedlegg, (\cref{app:risikoanalyse}). En fordel for gruppen var antall gruppemedlemmer, der det ble tatt ansvar for å sette seg inn og utføre arbeidsoppgaver ved fravær. Dette utspilte seg positivt, da flere fikk satt seg inn i andre deler av prosjektet.

  \section{Produkt}
  Før det ble bestemt hvilken løsning produktet skulle ha, ble brukerhistoriene vurdert for å se hva som passet best for kunden. Etter å ha vurdert både desktop-program i JavaFX og mobilapp, valgte vi å gå for en nettside. Hovedgrunnen er at en nettside er tilgjengelig både på mobil og datamaskin og krever ingen installasjon. Gruppen valgte nettrammeverket Django som utviklingsplattform. (Se vedlegg \cref{app:django} for beskrivelse av hvordan komponentene i Django fungerer.)

  I vår implementasjon rapporterer koiebrukere status på koiene ved å fylle ut forskjellige skjemaer og sende de inn. Dette innebærer vedstatus, vedlikeholdsbehov, forslag til tekniske utbedringer og reparasjoner for hver eneste koie. Brukere må fylle inn epost-en sin for å sende en rapport.

  Medlemmer av Koiestyret har mulighet til å logge seg inn på en adminside med oversikt over alle rapportene. Denne kan brukes til blant annet å planlegge veddugnader basert på vedprognoser eller reparasjoner. Kommunikasjon fra administrasjonen til koiebrukerne foregår via epost. (Se vedlegg \cref{app:brukermanual}  for en detaljert beskrivelse på hvordan produktet fungerer.)

  Løsningen kan testes på IP-adressen http://178.62.64.191/. (Se vedlegg \textit{Teknisk brukermanual} \cref{app:tekniskbrukermanual} for hvordan nettsiden kan settes opp på kundens egen server.)

  \section{Testing}

Når gruppen skulle teste produktet ble det valgt å ikke bruke unit tester i Python, men heller fokusere på å teste brukerhistoriene selv. Dette ble gjort i form av brukbarhetstester for kunden, og ved å skrive et testdesign dokument der hver brukerhistorie ble testet. Testdesignet består av hvordan testene ble gjennomført, og hvilke tilbakemeldinger er forventet ved gyldig/ugyldig inndata.

I tillegg har feilsøke modusen til Django blitt benyttet gjennom hele prosjektet for å sikre at koden fungerte for å få vist nettsiden. Feilsøke modusen blir benyttet ved å sette debug=True i settings.py filen, og tilbyr en detaljert visning av feilmeldingene som kan oppstå.

I Knibergs bok under kapittelet om testing\cite[side 128-131]{kniberg}, forklares viktigheten med testere i et \textit{Scrum} team. Det sies også at \textit{Scrum} teamet, sett bort fra \textit{Scrum}-masteren, skal være rolleløs og kryssfunksjonell. Hvor hele teamet bidrar med testingen, og samtidig har en som tar mer ansvar angående testing og kjører mer komplekse tester. Dette medfører at hele teamet blir involvert i testingen, som er positivt fordi hele teamet er ansvarlig for produktkvaliteten.

 \subsection{Brukbarhetstester}
Etter hver sprint ble brukbarhetstester gjennomført av kunden og bestod av å løse scenario oppgaver på nettsiden relatert til brukerhistoriene som ble fullført denne sprinten. Under testingen hadde to personer hovedansvaret for testen, disse dannet hovedrollene testleder og observatør. Testlederen hadde ansvaret for å hjelpe kunden gjennom oppgavene og at testen ble gjennomført. Observatørens oppgave var å observere hvordan kunden brukte produktet.

Via testene fikk gruppen sett hvordan produktet ble brukt av noen som ikke har jobbet med nettsiden. Dette hjalp å se hvor brukervennlig produktet var eller hvilke deler som var forvirrende for kunden. Flere av navnene under administrasjonssiden måtte endres, da disse var misvisende. Det samme gjaldt  linkene på hovedsiden.

\section{Konklusjon}
Vi endte opp med en velfungerende nettside der kravspesifikasjonene ble oppfylt innenfor satt frist. Gruppen klarte dette fordi vi kom tidlig igang og fokuserte sterkt på å jobbe effektivt og ikke sløse tid. Med \textit{Scrum}-modellen viste dette seg å være enklere enn først antatt. I avslutningsfasen av prosjektet hadde vi et retrospektiv-møte for hele prosjektperioden. (\cref{app:motereferat}) Her gjennomførte vi en postmortemanalyse for å finne ut hva vi ville ta med videre til fremtidige prosjekter, og hva vi ville unngå \cite[side 37]{dyba}. For eksempel kunne vi involvert flere i rapportskriving tidligere og rullert arbeidsoppgavene oftere. Gruppemedlemmene var enige om at taskboard, daglige Scrum-møter og faste møter/arbeidsøkter var velfungerende og viktig for at prosjektet gikk så bra. Vi hadde en god dialog innad i gruppen under hele prosjektet noe som var nøkkelen til en god gjennomføring. Uten bra samarbeid og god moral ville gruppen utført prosjektet på en mindre effektiv måte og resultatet blitt mindre polert. Gruppen har gjennom dette emnet lært mye om langsiktig jobbing med større prosjekter, og fått viktige verktøy som vi kan dra god nytte av i fremtiden med lignende problem.


\newpage
\phantomsection
\section{Referanser}
\bibliographystyle{agsm}
\bibliography{referanser}
\section{Vedlegg}

\phantomsection
%\appendix
%\begin{appendices}
%Testplan
\pagenumbering{gobble}
\includepdf[pages={1},pagecommand=\subsection{Testplan}\label{app:testplan}]{vedlegg/testplan}
\includepdf[pages={2-}]{vedlegg/testplan}

%Brukbarhetstest
\includepdf[pages={1},pagecommand=\subsection{Brukbarhetstest}\label{app:brukbarhetstest}]{vedlegg/brukbarhetstest}
\includepdf[pages={2-}]{vedlegg/brukbarhetstest}

%Brukermanual
\includepdf[pages={1},pagecommand={\subsection{Brukermanual}}\label{app:brukermanual}]{vedlegg/brukermanual}
\includepdf[pages={2-}]{vedlegg/brukermanual}

%Teknisk brukermanual
\includepdf[pages={1}, pagecommand={\subsection{Teknisk Brukermanual}}\label{app:tekniskbrukermanual}]{vedlegg/tekniskbrukermanual}
\includepdf[pages={2-}]{vedlegg/tekniskbrukermanual}

%Django
\includepdf[pages={1},pagecommand=\subsection{Django}\label{app:django}]{vedlegg/django}
\includepdf[pages={2-}]{vedlegg/django}

%Task-board
\includepdf[pages={1},pagecommand={\subsection{Taskboard}\label{app:taskboard}}]{vedlegg/taskboard}
\includepdf[pages={3-}]{vedlegg/taskboard}

% Risikoanalyse
\includepdf[pages={1},pagecommand=\subsection{Risikoanalyse}\label{app:risikoanalyse}]{vedlegg/risikoanal}
\includepdf[pages={2-}]{vedlegg/risikoanal}

%Motereferanse
\includepdf[pages={1},pagecommand=\subsection{Møtereferat 03.09.2015}\label{app:motereferat}]{vedlegg/motereferat}
\includepdf[pages={2-}]{vedlegg/motereferat}

%Retrospektiv
\subsection{Retrospektiv}\nobreak
    %Todo: Skrive en liten innledning/ forklaring til retrospektiv-sprint...


%Sprint 1
\includepdf[pages={1},pagecommand={\subsubsection{Sprint 1}\label{app:sprint1}}]{vedlegg/sprint1}
\includepdf[pages={2-}]{vedlegg/sprint1}

%Sprint 2
\includepdf[pages={1},pagecommand={\subsubsection{Sprint 2}\label{app:sprint2}}]{vedlegg/sprint2}
\includepdf[pages={2-}]{vedlegg/sprint2}

%Sprint 3
\includepdf[pages={1},pagecommand=\subsubsection{Sprint 3}\label{app:sprint3}]{vedlegg/sprint3}
\includepdf[pages={2-}]{vedlegg/sprint3}

%Sprint 4
\includepdf[pages={1},pagecommand=\subsubsection{Sprint 4}\label{app:sprint4}]{vedlegg/sprint4}
\includepdf[pages={2-}]{vedlegg/sprint4}

%%RELEASEPLAN
%\subsection{Releaseplan}

%¿Hvorfor er disse kommentert ut?
%\includepdf[pages={1},pagecommand=\subsubsection{Releaseplan-v1}\label{app:releaseP1}]{vedlegg/releaseP1}
%\includepdf[pages={2-}]{vedlegg/releaseP1}

%\includepdf[pages={1},pagecommand=\subsubsection{Releaseplan-v2}\label{app:releaseP2}]{vedlegg/releaseP2}
%\includepdf[pages={-}]{vedlegg/releaseP2}

\includepdf[pages={1},pagecommand=\subsection{Releaseplan-v3}\label{app:releaseP3}]{vedlegg/releaseP3}
\includepdf[pages={2-}]{vedlegg/releaseP3}

%%Burndown Charts
\includepdf[pages={1},pagecommand={\subsection{Brenndiagrammer}\label{app:brenn}}]{vedlegg/burndowncharts}
\includepdf[pages={2-}]{vedlegg/burndowncharts}

%\end{appendices}
\end{document}
